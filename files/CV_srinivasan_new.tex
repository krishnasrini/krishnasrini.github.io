%%%%%%%%%%%%%%%%%%%%%%%%%%%%%%%%%%%%%%%%%
% Medium Length Professional CV
% LaTeX Template
% Version 2.0 (8/5/13)
%
% This template has been downloaded from:
% http://www.LaTeXTemplates.com
%
% Original author:
% Trey Hunner (http://www.treyhunner.com/)
%
% Important note:
% This template requires the resume.cls file to be in the same directory as the
% .tex file. The resume.cls file provides the resume style used for structuring the
% document.
%
%%%%%%%%%%%%%%%%%%%%%%%%%%%%%%%%%%%%%%%%%

%----------------------------------------------------------------------------------------
%	PACKAGES AND OTHER DOCUMENT CONFIGURATIONS
%----------------------------------------------------------------------------------------

\documentclass{resume} % Use the custom resume.cls style
%\usepackage[left=0.75in,top=0.6in,right=0.75in,bottom=0.6in]{geometry} % Document margins
% \usepackage[a4paper, left=1in,top=0.8in,right=0.8in,bottom=0.9in]{geometry}
\usepackage[a4paper, left=0.76in,top=0.80in,right=0.76in,bottom=0.9in]{geometry}


\newcommand{\tab}[1]{\hspace{.2667\textwidth}\rlap{#1}}
\newcommand{\itab}[1]{\hspace{0em}\rlap{#1}}
\usepackage{fancyhdr}
\pagestyle{fancy}
\usepackage{eurosym}
\usepackage{eurosym}
\usepackage{array}
\usepackage{changepage}

 \renewcommand{\headrulewidth}{0pt}

\begin{document}

% \begin{center}
%   \textbf{\MakeUppercase{krishna srinivasan}} \\
%   \href{https://www.krishnasrini.com}{https://krishnasrini.com} \\
%   \href{krishna.srinivasan@unidistance.ch}{krishna.srinivasan@unidistance.ch}
% \end{center}

\vspace*{-1cm}
\begin{center}
  \textbf{\MakeUppercase{Krishna Srinivasan}} \\
\end{center}
\vspace*{0.7cm}

\begin{rSection}{Contact Information}
  \begin{tabular}{ @{} >{}l @{\hspace{19.5ex}} l }
    UniDistance Suisse &  \href{https://www.krishnasrini.com}{https://krishnasrini.com} \\
    Schinerstrasse 18 &  \href{krishna.srinivasan@unidistance.ch}{krishna.srinivasan@unidistance.ch}
    \\
    3900 Brig-Glis, Switzerland & +41 791529966 \\
  \end{tabular}
\end{rSection}

%%%%%%%%%%%%%%%%%%%%%%%%%%%%%%%%%%%%%%%%%
\begin{rSection}{References}

  \begin{tabular}{ @{} >{}l @{\hspace{15ex}} l }

  Ernst Fehr & Christopher Roth  \\
  University of Zurich & University of Cologne   \\   \vspace{0.3cm}
  ernst.fehr@econ.uzh.ch & roth@wiso.uni-koeln.de  \\ 
  Dmitry Taubinsky & Bj\"{o}rn Bartling     \\
  University of California - Berkeley & University of Zurich  \\  
  dmitry.taubinsky@berkeley.edu & bjoern.bartling@econ.uzh.ch  \\

  \end{tabular}
\end{rSection}

%%%%%%%%%%%%%%%%%%%%%%%%%%%%%%%%%%%%%%%%%
\begin{rSection}{Professional Affiliations}
  \begin{tabular}{ @{} p{0.8\linewidth} >{\raggedleft\arraybackslash}p{0.18\linewidth} }
  Postdoctoral Researcher, UniDistance Suisse &  2024 - \phantom{2024} \\
  \hspace{0.5cm} Advisor: Manuel Grieder \\
  % Visiting Fellow, University of Zurich & 2024 - \phantom{2024}    \\
  Visiting Scholar, The Wharton School & 2025, Fall \phantom{e} \\
  \hspace{0.5cm} Host: Benjamin Lockwood 
\end{tabular}
\end{rSection}


%%%%%%%%%%%%%%%%%%%%%%%%%%%%%%%%%%%%%%%%%
\begin{rSection}{Education}
  \begin{tabular}{ @{} p{0.8\linewidth} >{\raggedleft\arraybackslash}p{0.18\linewidth} }
  Ph.D. Economics, University of Zurich  &  2018 - 2024 \\
  M.Sc. Econometrics and Mathematical Economics, Tilburg University & 2016 - 2017  \\
  B.Sc. Economics, Mathematics, Statistics, Christ University & 2013 - 2016
  \end{tabular}
\end{rSection}


%%%%%%%%%%%%%%%%%%%%%%%%%%%%%%%%%%%%%%%%%
\begin{rSection}{Research Visits}
  \begin{tabular}{ @{} p{0.78\linewidth} >{\raggedleft\arraybackslash}p{0.2\linewidth} }
  University of Zurich  & 08/2024 - 08/2025 \\
  Norwegian School of Economics (Host: Bertil Tungodden) & 08/2022 - 01/2023  \\
  University of California - Berkeley (Host: Dmitry Taubinsky) &  08/2021 - 12/2021 \\
  \end{tabular}
\end{rSection}

%%%%%%%%%%%%%%%%%%%%%%%%%%%%%%%%%%%%%%%%%
\begin{rSection}{Research and Teaching Fields}
  Primary Fields: Public Economics, Behavioral Economics  \\
  Secondary Fields: Experimental Economics, Political Economy
\end{rSection}

%%%%%%%%%%%%%%%%%%%%%%%%%%%%%%%%%
\begin{rSection}{Job Market Papers}
  JMP1 ``Preferences for Taxing Personal Characteristics'' (with Julien Senn)

Tagging—conditioning taxes on income and personal characteristics correlated with earning ability—can improve efficiency. However, its social acceptability remains uncertain, as it may conflict with horizontal equity, the principle that ``equals'' should be treated equally. We study citizens' support for tagging in an online vignette experiment with the U.S. general population ($N=3012$). Our design manipulates three theoretically relevant features of characteristics: immutability, correlation with ability, and correlation with needs. Characteristics correlated with ability and needs receive higher support, consistent with theoretical prescriptions, while immutability does not significantly predict support, contrary to theory. Fairness concerns, capturing both horizontal and vertical equity, are the strongest predictor of support; other concerns also matter, but less so. We incorporate fairness into our theoretical model, which can limit or amplify tagging relative to the canonical utilitarian model. Finally, we synthesize insights from theory, our experiment, and practice.

  JMP2:   ``Who Should Get Money? Estimating Welfare Weights in the U.S.'' (with Francesco Capozza). Reject and Resubmit, \textbf{American Economic Review} (resubmitted). \\ \vspace{0.6em}
  { \normalsize  \hspace*{0.6em} Distinguished CESifo Affiliate Award in Public Economics 2024 (awarded to Francesco Capozza)} 

  Evaluating the desirability of a reform typically involves weighing the gains of winners against the losses of losers using welfare weights, which measure the value that society places on a \$1 increase in an individual's consumption. We elicit the welfare weights of the U.S. general population using experiments and show their robustness, validity, and temporal stability. We estimate an income elasticity of welfare weights between $-0.78$ and $-0.70$, which is roughly 5-9 times more progressive than the weights implied by U.S. tax and transfer policies. We use the estimated welfare weights to derive optimal income taxes.
\end{rSection}

\begin{rSection}{Working Papers}

  % ``Preferences for Taxing Personal Characteristics'' (with Julien Senn)


  ``Fair (P)redistribution'' (with with Justin Valasek and Weijia Wang). Reject and Resubmit, \textbf{Journal of Public Economics.}

  ``Science by Consensus: Eliciting Citizens' and Experts' R\&D Spending Priorities'' (with Francesco Capozza and Mattie Toma) 

  ``Paternalism: Determinants of Demand and Supply'' (with Bj\"{o}rn Bartling)

  %   We elicit the welfare weights assigned by the general population of the U.S. using a real-stakes experiment. Welfare weights measure the relative social value of providing individuals with an additional dollar of consumption. These weights can be directly used to evaluate the desirability of reforms in domains such as taxation and cash transfers. We find that the general population weights are more progressive than the weights implied by tax and transfer policies in the U.S., indicating that the general population desires additional redistribution. The general population weights are less progressive than those frequently used in the literature. We explore the implications of these weights for optimal income taxes.

\end{rSection}

\begin{rSection}{Work in Progress}

  ``Understanding Administrative Burden: A Field Experiment on the Hassle Costs of Tax Filing in Germany'' (with Manuel Grieder and Michael Kurschilgen)

  ``The Externality Atlas'' (with Max M\"{u}ller and Christopher Roth)
  
  ``What Motivates Censorship?'' (with George Beknazar-Yuzbashev, Francesco Capozza, and Mateusz Stalinski)

  ``Eliciting Welfare Weights Based on Socio-Economic Characteristics'' (with Katy Bergstrom, Francesco Capozza, William Dodds, and Juan Rios)
\end{rSection}

%%%%%%%%%%%%%%%%%%%%%%%%%%%%%%%%%%%%%%%%%
\begin{rSection}{Research Grants}
  \begin{tabular}{ @{} p{0.88\linewidth} >{\raggedleft\arraybackslash}p{0.10\linewidth} }

  Diligentia Project Grant (with Manuel Grieder and Michael Kurschilgen): \euro33,248 & 2025 \\
  J-PAL Science for Progress Initiative (with Francesco Capozza and Mattie Toma): \$23,423 & 2025 \\
  UZH GRC Short Grant (with Julien Senn): CHF~5000 & 2023 \\
  UZH URPP `Equality of Opportunity' Grant (with Francesco Capozza): CHF~8,680 & 2022 \\
  Russell Sage Foundation Small Grants (with Francesco Capozza): \$7,850 & 2018\\
  UZH Director's Grant (with Francesco Capozza): CHF 1,500 & 2021
  \end{tabular}
\end{rSection}

%%%%%%%%%%%%%%%%%%%%%%%%%%%%%%%%%

\begin{rSection}{Scholarships}
  \begin{tabular}{ @{} p{0.8\linewidth} >{\raggedleft\arraybackslash}p{0.18\linewidth} }
  UZH Doc.Mobility &  2022\\
  UZH Departmental scholarship for Ph.D. students & 2018 - 2024
  \end{tabular}
\end{rSection}



%%%%%%%%%%%%%%%%%%%%%%%%%%%%%%%%%%%%%%%%%
\begin{rSection}{Teaching Experience}

  \textbf{Teaching Assistant at University of Zurich}

    \begin{tabular}{ @{} p{0.8\linewidth} >{\raggedleft\arraybackslash}p{0.18\linewidth} }
    Behavioral Public Economics (master's) & 2022 \\
    Behavioral Economics (bachelor's) & 2021 \\
    Introduction to Behavioral Economics (master's) & 2020, 2021 \\
    The Economics of Paternalism (master's) & 2020 \\
    Behavioral Theory (master's) & 2019, 2020
    \end{tabular}

    \newpage 
    \textbf{Tutorship at MAK - Tilburg University}

    \begin{tabular}{ @{} p{0.8\linewidth} >{\raggedleft\arraybackslash}p{0.18\linewidth} }
      Statistics 2 for International Business Administration (master's) & 2016, 2017 \\
    \end{tabular}

  \end{rSection}

%%%%%%%%%%%%%%%%%%%%%%%%%%%%%%%%%
\begin{rSection}{Professional Activities}
  Refereeing: Journal of Political Economy - Micro, Journal of Public Economics, Journal of the European Economic Association, Oxford Bulletin of Economics and Statistics, Oxford Economic Papers
\end{rSection}


%%%%%%%%%%%%%%%%%%%%%%%%%%%%%%%%%%%%%%%%%

\begin{rSection}{Invited Seminars, Conferences, and Summer Schools (including scheduled)}

  \textbf{Invited Seminars:} \textbf{2025:} Tulane University, Helmut Schmidt University -- Hamburg, University of Cologne (C-SEB) \textbf{2023:} M.S. Ramaiah University of Applied Sciences

  \textbf{Conferences:} \textbf{2025:} Lofoten International Symposium on Inequality and Taxation, Ludwig Erhard ifo Conference on Institutional Economics, CESifo Area Conference on Public Economics \textbf{2024:}  Verein f\"ur Socialpolitik, ESA World Meeting (Bogota), URPP Taxation \& Inequality Conference (poster) \textbf{2023:}
    Zurich Workshop in Economics and Psychology, EEA-ESEM Congress, Annual Congress of the International Institute of Public Finance (IIPF),  Ludwig Erhard ifo Conference on Institutional Economics,  Young Swiss Economist Meeting \textbf{2022:} BREW-ESA,  Spring School in Behavioral Economics (poster), Fairness and the Moral Mind Workshop, In\_equality Conference

  \textbf{Summer Schools:} \textbf{2025:} Chicago School of Experimental Economics \textbf{2023:} briq Summer School in Behavioral Economics 
   \textbf{2022:} Spring School in Behavioral Economics, HCEO-FAIR Summer School on Socioeconomic Inequality 

\end{rSection}

%%%%%%%%%%%%%%%%%%%%%%%%%%%%%%%%%%%%%%%%%
\begin{rSection}{Personal Details}
  \begin{tabular}{ @{} >{}l @{\hspace{3.5ex}} l }
    Date of birth & 04 December 1994 \\
    Citizenship & Indian \\
    Languages & English (Native)
  \end{tabular}
\end{rSection}


\end{document}

