%%%%%%%%%%%%%%%%%%%%%%%%%%%%%%%%%%%%%%%%%
% Medium Length Professional CV
% LaTeX Template
% Version 2.0 (8/5/13)
%
% This template has been downloaded from:
% http://www.LaTeXTemplates.com
%
% Original author:
% Trey Hunner (http://www.treyhunner.com/)
%
% Important note:
% This template requires the resume.cls file to be in the same directory as the
% .tex file. The resume.cls file provides the resume style used for structuring the
% document.
%
%%%%%%%%%%%%%%%%%%%%%%%%%%%%%%%%%%%%%%%%%

%----------------------------------------------------------------------------------------
%	PACKAGES AND OTHER DOCUMENT CONFIGURATIONS
%----------------------------------------------------------------------------------------

\documentclass{resume} % Use the custom resume.cls style
%\usepackage[left=0.75in,top=0.6in,right=0.75in,bottom=0.6in]{geometry} % Document margins
\usepackage[left=0.64in,top=0.7in,right=0.75in,bottom=0.75in]{geometry}
\newcommand{\tab}[1]{\hspace{.2667\textwidth}\rlap{#1}}
\newcommand{\itab}[1]{\hspace{0em}\rlap{#1}}
\usepackage{fancyhdr}
\pagestyle{fancy}
\usepackage{eurosym}
\usepackage{eurosym}
\usepackage{array}
\usepackage{changepage}
 
 \renewcommand{\headrulewidth}{0pt}

\begin{document}

% \begin{flushright}
%   \textbf{\MakeUppercase{krishna srinivasan}} \\
%   \href{https://www.krishnasrini.com}{https://krishnasrini.com} \\
%   \href{krishna.srinivasan@econ.uzh.ch}{krishna.srinivasan@econ.uzh.ch}   
% \end{flushright}

\vspace*{-0.8cm}

\begin{center}
  \textbf{\MakeUppercase{Krishna Srinivasan}}
\end{center}

% \hrule
\begin{rSection}{Contact Information}
  \begin{tabular}{ @{} >{}l @{\hspace{20ex}} l }
    University of Zurich &   \href{https://www.krishnasrini.com}{https://krishnasrini.com} \\
    Bluemlisalpstrasse 10 &   \href{krishna.srinivasan@econ.uzh.ch}{krishna.srinivasan@econ.uzh.ch}   \\
    8006 Zurich, Switzerland\\
  \end{tabular}
\end{rSection}


%%%%%%%%%%%%%%%%%%%%%%%%%%%%%%%%%%%%%%%%%
\begin{rSection}{References}

  \begin{tabular}{ @{} >{}l @{\hspace{15ex}} l }
  
  Björn Bartling  & Ernst Fehr  \\
  University of Zurich & University of Zurich \\  \vspace*{0.3cm}
  bjoern.bartling@econ.uzh.ch & ernst.fehr@econ.uzh.ch\\
  
  Michel Marechal & Bertil Tungodden \\ 
  University of Zurich & Norwegian School of Economics \\
  michel.marechal@econ.uzh.ch & bertil.tungodden@nhh.no \\
  \end{tabular}
  \end{rSection}
  
  
  \begin{rSection}{Placement Director} 
  \begin{tabular}{ @{} >{}l @{\hspace{13ex}} l }
    Joachim Voth \\
    University of Zurich \\
    voth@econ.uzh.ch
  \end{tabular}  
\end{rSection}

%%%%%%%%%%%%%%%%%%%%%%%%%%%%%%%%%%%%%%%%%
\begin{rSection}{Education}
  \begin{tabular}{ @{} p{0.8\linewidth} >{\raggedleft\arraybackslash}p{0.18\linewidth} }
  Ph.D. Economics, University of Zurich  &  08/2018 - present \\
  \hspace*{1em} Expected Completion Date: 04/2024 \\
  M.Sc. Econometrics and Mathematical Economics, Tilburg University & 2016 - 2017  \\
  B.Sc. Economics, Mathematics, Statistics, Christ University & 2013 - 2016 
  \end{tabular}
\end{rSection}


%%%%%%%%%%%%%%%%%%%%%%%%%%%%%%%%%%%%%%%%%
\begin{rSection}{Research Visits} 
  \begin{tabular}{ @{} p{0.8\linewidth} >{\raggedleft\arraybackslash}p{0.18\linewidth} }
  Visiting Student, Norwegian School of Economics & 08/2022 - 01/2023  \\
  Visiting Student, University of California - Berkeley &  08/2021 - 12/2021
  \end{tabular}
\end{rSection}

%%%%%%%%%%%%%%%%%%%%%%%%%%%%%%%%%%%%%%%%%
\begin{rSection}{Research and Teaching Fields}
  Primary fields: Behavioral Economics, Public economics \\
  Secondary fields: Experimental Economics, Political Economy
\end{rSection}


%%%%%%%%%%%%%%%%%%%%%%%%%%%%%%%%%
\begin{rSection}{Job Market Paper}
  % \begin{tabular}{p{15cm}}
    ``Who Should Get Money? Estimating Welfare Weights in the U.S.'' (with Francesco Capozza) 

    Evaluating the desirability of a reform commonly involves weighing the gains of the winners against the losses of the losers using welfare weights. Welfare weights measure the relative social value of providing individuals with an additional dollar of consumption. These weights can capture various normative ideals, such as utilitarianism or libertarianism. We elicit the welfare weights assigned by the general population of the U.S. using a real-stakes experiment. The general population weights can be used to identify socially acceptable policies. We find that the general population weights are more progressive than the weights implied by tax and transfer policies in the U.S., indicating that the general population desires additional redistribution. The general population weights are less progressive than the weights frequently used in the optimal policy literature. We explore the implications of the general population weights by calibrating the optimal non-linear income taxes.   
    % \end{tabular}
\end{rSection}


%%%%%%%%%%%%%%%%%%%%%%%%%%%%%%%%%
\begin{rSection}{Work in Progress}
  
  ``Who Deserves a Tax Break and Why? Evidence on
   Preferences for Taxing Personal Characteristics'' (with Julien Senn) 

  ``Paternalism: Determinants of Demand and Supply'' (with Bj\"{o}rn Bartling)

  ``Eliciting Funding Priorities Using Domain-Specific Welfare Weights'' (with Francesco Capozza and Morten St{\o}stad)

  ``What Motivates Censorship?'' (with Francesco Capozza)
\end{rSection}


%%%%%%%%%%%%%%%%%%%%%%%%%%%%%%%%%%%%%%%%%
\begin{rSection}{Research Grants}
  \begin{tabular}{ @{} p{0.9\linewidth} >{\raggedleft\arraybackslash}p{0.08\linewidth} }

  UZH GRC Short Grant (with Julien Senn) - CHF 5000 & 2023 \\
  UZH URPP Equality of Opportunity Grant (with Francesco Capozza) - CHF 8680 & 2022 \\
  Russel Sage Foundation Small Grants (with Francesco Capozza) - \$7850 & 2018\\
  UZH Director's Grant (with Francesco Capozza) - CHF 1500 & 2021 
  \end{tabular}
\end{rSection}

%%%%%%%%%%%%%%%%%%%%%%%%%%%%%%%%%
\begin{rSection}{Scholarships:}
  \begin{tabular}{ @{} p{0.8\linewidth} >{\raggedleft\arraybackslash}p{0.18\linewidth} }
  UZH Doc.Mobility &  2022\\
  Departmental scholarship for Ph.D. students & 2018 - 2024
  \end{tabular}
\end{rSection}

%%%%%%%%%%%%%%%%%%%%%%%%%%%%%%%%%%%%%%%%%
\begin{rSection}{Teaching Experience}
  \textbf{Teaching Assistant at University of Zurich} 

    \begin{tabular}{ @{} p{0.8\linewidth} >{\raggedleft\arraybackslash}p{0.18\linewidth} }
    Behavioral Public Economics (master's) & 2022 \\
    Behavioral Economics (bachelor's) & 2021 \\
    Introduction to Behavioral Economics (master's) & 2020, 2021 \\
    The Economics of Paternalism (master's) & 2020 \\
    Behavioral Theory (master's) & 2019, 2020 
    \end{tabular}
  \end{rSection}
  
%%%%%%%%%%%%%%%%%%%%%%%%%%%%%%%%%
\begin{rSection}{Professional Activities}
  Refereeing: Journal of Public Economics
\end{rSection}

%%%%%%%%%%%%%%%%%%%%%%%%%%%%%%%%%%%%%%%%%
\begin{rSection}{Presentations}
  \begin{tabular}{ @{} p{0.9\linewidth} >{\raggedleft\arraybackslash}p{0.08\linewidth} }
    EEA Conference, Annual Congress of the IIPF, Ramaiah University of Applied Sciences, Ludwig Erhard ifo Conference on Institutional Economics,  Young Swiss Economist Meeting & 2023 \\ 

    BREW-ESA,  Spring School in Behavioral Economics (Poster), Fairness and the Moral Mind Workshop, In\_equality Conference &   2022   
  \end{tabular}
\end{rSection}

%%%%%%%%%%%%%%%%%%%%%%%%%%%%%%%%%%%%%%%%%
\begin{rSection}{Summer Schools}
  \begin{tabular}{ @{} p{0.9\linewidth}>{\raggedleft\arraybackslash}p{0.08\linewidth} }
    briq Summer School in Behavioral Economics, & 2023 \\
    Spring School in Behavioral Economics, HCEO-FAIR Summer School on Socioeconomic Inequality & 2022  
  \end{tabular}
\end{rSection}

%%%%%%%%%%%%%%%%%%%%%%%%%%%%%%%%%%%%%%%%%
\begin{rSection}{Personal Details}
  \begin{tabular}{ @{} >{}l @{\hspace{3.5ex}} l }
    Date of birth & 04 December 1994 \\
    Citizenship & Indian \\
    Languages & English (Native)
  \end{tabular}
\end{rSection}


%%%%%%%%%%%%%%%%%%%%%%%%%%%%%%%%%%%%%%%%%
% \begin{rSection}{References}

% \begin{tabular}{ @{} >{}l @{\hspace{13ex}} l }

% Björn Bartling  & Ernst Fehr  \\
% University of Zurich & University of Zurich \\  \vspace*{0.3cm}
% bjoern.bartling@econ.uzh.ch & ernst.fehr@econ.uzh.ch\\

% Michel Marechal & Bertil Tungodden \\ 
% University of Zurich & Norwegian School of Economics \\
% michel.marechal@econ.uzh.ch & bertil.tungodden@nhh.no \\
% \end{tabular}
% \end{rSection}


% \begin{rSection}{Placement Director} 
% \begin{tabular}{ @{} >{}l @{\hspace{13ex}} l }
%   Joachim Voth \\
%   University of Zurich \\
%   voth@econ.uzh.ch
% \end{tabular}  
% \end{rSection}


\end{document}

