%%%%%%%%%%%%%%%%%%%%%%%%%%%%%%%%%%%%%%%%%
% Medium Length Professional CV
% LaTeX Template
% Version 2.0 (8/5/13)
%
% This template has been downloaded from:
% http://www.LaTeXTemplates.com
%
% Original author:
% Trey Hunner (http://www.treyhunner.com/)
%
% Important note:
% This template requires the resume.cls file to be in the same directory as the
% .tex file. The resume.cls file provides the resume style used for structuring the
% document.
%
%%%%%%%%%%%%%%%%%%%%%%%%%%%%%%%%%%%%%%%%%

%----------------------------------------------------------------------------------------
%	PACKAGES AND OTHER DOCUMENT CONFIGURATIONS
%----------------------------------------------------------------------------------------

\documentclass{resume} % Use the custom resume.cls style
%\usepackage[left=0.75in,top=0.6in,right=0.75in,bottom=0.6in]{geometry} % Document margins
\usepackage[a4paper, left=1in,top=0.9in,right=1in,bottom=0.9in]{geometry}
\newcommand{\tab}[1]{\hspace{.2667\textwidth}\rlap{#1}}
\newcommand{\itab}[1]{\hspace{0em}\rlap{#1}}
\usepackage{fancyhdr}
\pagestyle{fancy}
\usepackage{eurosym}
\usepackage{eurosym}
\usepackage{array}
\usepackage{changepage}

 \renewcommand{\headrulewidth}{0pt}

\begin{document}

\begin{center}
  \textbf{\MakeUppercase{krishna srinivasan}} \\
  \href{https://www.krishnasrini.com}{https://krishnasrini.com} \\
  \href{krishna.srinivasan@unidistance.ch}{krishna.srinivasan@unidistance.ch}
\end{center}

% \vspace*{-1cm}
% \begin{center}
%   \textbf{\MakeUppercase{Krishna Srinivasan}} \\
% \end{center}
% \vspace*{0.7cm}

% % \hrule
% \begin{rSection}{Contact Information}
%   \begin{tabular}{ @{} >{}l @{\hspace{19.5ex}} l }
%     University of Zurich &  \href{https://www.krishnasrini.com}{https://krishnasrini.com} \\
%     Bluemlisalpstrasse 10 &  \href{krishna.srinivasan@econ.uzh.ch}{krishna.srinivasan@econ.uzh.ch}   \\
%     8006 Zurich, Switzerland \\
%   \end{tabular}
% \end{rSection}

\vspace*{1cm}
%%%%%%%%%%%%%%%%%%%%%%%%%%%%%%%%%%%%%%%%%
\begin{rSection}{Professional Affiliations}
  \begin{tabular}{ @{} p{0.8\linewidth} >{\raggedleft\arraybackslash}p{0.18\linewidth} }
  Postdoctoral Researcher, UniDistance Suisse &  2024 - \phantom{2024} \\
  Visiting Fellow, University of Zurich & 2024 - \phantom{2024}    \\
\end{tabular}
\end{rSection}


%%%%%%%%%%%%%%%%%%%%%%%%%%%%%%%%%%%%%%%%%
\begin{rSection}{Education}
  \begin{tabular}{ @{} p{0.8\linewidth} >{\raggedleft\arraybackslash}p{0.18\linewidth} }
  Ph.D. Economics, University of Zurich  &  2018 - 2024 \\
  M.Sc. Econometrics and Mathematical Economics, Tilburg University & 2016 - 2017  \\
  B.Sc. Economics, Mathematics, Statistics, Christ University & 2013 - 2016
  \end{tabular}
\end{rSection}


%%%%%%%%%%%%%%%%%%%%%%%%%%%%%%%%%%%%%%%%%
\begin{rSection}{Research Visits}
  \begin{tabular}{ @{} p{0.78\linewidth} >{\raggedleft\arraybackslash}p{0.2\linewidth} }
    Norwegian School of Economics (Host: Bertil Tungodden)  & Fall 2022  \\
  %   Visiting Student, Norwegian School of Economics  & Fall 2022  \\
  % \hspace*{1em} Host: Bertil Tungodden \\
  University of California - Berkeley (Host: Dmitry Taubinsky) &  Fall 2021 \\
  % Visiting Student, University of California - Berkeley &  Fall 2021 \\
  % \hspace*{1em} Host: Dmitry Taubinsky \\
  \end{tabular}
\end{rSection}

%%%%%%%%%%%%%%%%%%%%%%%%%%%%%%%%%%%%%%%%%
\begin{rSection}{Research and Teaching Fields}
  Primary fields: Behavioral Economics, Public economics \\
  Secondary fields: Experimental Economics, Political Economy
\end{rSection}

%%%%%%%%%%%%%%%%%%%%%%%%%%%%%%%%%
\begin{rSection}{Working Papers}
  ``Who Should Get Money? Estimating Welfare Weights in the U.S.'' (with Francesco Capozza) \\ \vspace{0.1em}
  { \normalsize  \hspace*{0.9em} Reject and Resubmit at the American Economic Review }\\\vspace{0.1em}
  { \normalsize  \hspace*{0.8em} Distinguished CESifo Affiliate Award for best paper in Public Economics 2024 (awarded \hspace*{1.3em} to Francesco Capozza)} 

  ``Who Deserves a Tax Break and Why? Evidence on
  Preferences for Taxing Personal Characteristics'' (with Julien Senn)

  ``Paternalism: Determinants of Demand and Supply'' (with Bj\"{o}rn Bartling)

  %   We elicit the welfare weights assigned by the general population of the U.S. using a real-stakes experiment. Welfare weights measure the relative social value of providing individuals with an additional dollar of consumption. These weights can be directly used to evaluate the desirability of reforms in domains such as taxation and cash transfers. We find that the general population weights are more progressive than the weights implied by tax and transfer policies in the U.S., indicating that the general population desires additional redistribution. The general population weights are less progressive than those frequently used in the literature. We explore the implications of these weights for optimal income taxes.
  \end{rSection}

%%%%%%%%%%%%%%%%%%%%%%%%%%%%%%%%%
\begin{rSection}{Work in Progress}

  ``Science by Consensus: Eliciting Citizens' R\&D Funding Priorities'' (with Francesco Capozza and Mattie Toma) 

  % ``Eliciting Funding Priorities Using Domain-Specific Welfare Weights'' (with Francesco Capozza and Morten St{\o}stad)

  % ``What Motivates Censorship?'' (with George Beknazar-Yuzbashev, Francesco Capozza, and Mateusz Stalinski)
\end{rSection}

%%%%%%%%%%%%%%%%%%%%%%%%%%%%%%%%%%%%%%%%%
\begin{rSection}{Research Grants}
  \begin{tabular}{ @{} p{0.82\linewidth} >{\raggedleft\arraybackslash}p{0.16\linewidth} }

  UZH GRC Short Grant (with Julien Senn) - CHF 5000 & 2023 \\
  UZH URPP `Equality of Opportunity' Project Grant (with Francesco Capozza) - CHF 8680 & 2022 \\
  Russel Sage Foundation Small Grants (with Francesco Capozza) - \$7850 & 2018\\
  UZH Director's Grant (with Francesco Capozza) - CHF 1500 & 2021
  \end{tabular}
\end{rSection}

%%%%%%%%%%%%%%%%%%%%%%%%%%%%%%%%%
\newpage

\begin{rSection}{Scholarships}
  \begin{tabular}{ @{} p{0.8\linewidth} >{\raggedleft\arraybackslash}p{0.18\linewidth} }
  UZH Doc.Mobility &  2022\\
  UZH Departmental scholarship for Ph.D. students & 2018 - 2024
  \end{tabular}
\end{rSection}

%%%%%%%%%%%%%%%%%%%%%%%%%%%%%%%%%
\begin{rSection}{Professional Activities}
  Refereeing: Journal of the European Economic Association, Journal of Public Economics, Oxford Economic Papers
\end{rSection}


%%%%%%%%%%%%%%%%%%%%%%%%%%%%%%%%%%%%%%%%%
\begin{rSection}{Teaching Experience}

  \textbf{Teaching Assistant at University of Zurich}

    \begin{tabular}{ @{} p{0.8\linewidth} >{\raggedleft\arraybackslash}p{0.18\linewidth} }
    Behavioral Public Economics (master's) & 2022 \\
    Behavioral Economics (bachelor's) & 2021 \\
    Introduction to Behavioral Economics (master's) & 2020, 2021 \\
    The Economics of Paternalism (master's) & 2020 \\
    Behavioral Theory (master's) & 2019, 2020
    \end{tabular}

    \textbf{Tutorship at MAK - Tilburg Univeristy}

    \begin{tabular}{ @{} p{0.8\linewidth} >{\raggedleft\arraybackslash}p{0.18\linewidth} }
      Statistics 2 for International Business Administration (master's) & 2016, 2017 \\
    \end{tabular}

  \end{rSection}



%%%%%%%%%%%%%%%%%%%%%%%%%%%%%%%%%%%%%%%%%
\begin{rSection}{Conferences and Inivited Seminars (including scheduled)}
    \textbf{2025:} Lofoten International Symposium on Inequality and Taxation (scheduled), Ludwig Erhard ifo Conference on Institutional Economics (scheduled), CESifo Area Conference on Public Economics (scheduled) \textbf{2024:}  Verein F\"ur Socialpolitik, ESA World Meeting Bogota, URPP Taxation \& Inequality Conference (poster) \textbf{2023:}
    Zurich Workshop in Economics and Psychology, EEA-ESEM Congress, Annual Congress of the IIPF, Ramaiah University of Applied Sciences, Ludwig Erhard ifo Conference on Institutional Economics,  Young Swiss Economist Meeting \textbf{2022:} BREW-ESA,  Spring School in Behavioral Economics (Poster), Fairness and the Moral Mind Workshop, In\_equality Conference
\end{rSection}

%%%%%%%%%%%%%%%%%%%%%%%%%%%%%%%%%%%%%%%%%
\begin{rSection}{Summer Schools}
  \textbf{2023:} briq Summer School in Behavioral Economics 
  \textbf{2022:} Spring School in Behavioral Economics, HCEO-FAIR Summer School on Socioeconomic Inequality 
\end{rSection}

%%%%%%%%%%%%%%%%%%%%%%%%%%%%%%%%%%%%%%%%%
\begin{rSection}{Personal Details}
  \begin{tabular}{ @{} >{}l @{\hspace{3.5ex}} l }
    Date of birth & 04 December 1994 \\
    Citizenship & Indian \\
    Languages & English (Native)
  \end{tabular}
\end{rSection}


\end{document}

